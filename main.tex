\begin{proof}
	A_h=A_{0h}+\diag(c_i) où A_{0h}=\mqty(\frac2{h^2}&-\frac1{h^2}&0&0&\cdots&0\\-\frac1{h^2}&\frac2{h^2}&1\frac1{h^2}0&\cdots&0)\\0&-\frac1{h^2}&\frac2{h^2}&-\frac1{h^2}&\cdots&0\\0&\dots&\dots&0&0&\frac2{h^2}&-\frac1{h^2}\\0&...&0&-\frac1{h^2}&\frac2{h^2})

\diag(c_i)=\mdiag{c_1,c_2...c_{N-1},c_N}

A+h inversible et A_{0h} inversible (A_{0h}=A_h dans le cas c_i =0 \forall i)

On a: A_{0h}\dmo-A_h\dmo=A_{0h}\dmo A_h A_h\dmo -A_{0h}\dmoA_{0h}A_h\dmo=A_{0h}\dmo(A_h-A_{0h})A_h\dmo

c_i≥0 \forall i => A_h-A_{0h}=\diag(c_i)≥0

De plus: A_h et A_{0h} monotones => A_h\dmo≥0 et A\dmo_{0h}≥0. D'où : A_{0h}\dmo≥A_h\dmo≥0. On va montrer que \norm{A_{0h}\dmo}_∞≤\frac18 et donc \norm{A_h\dmo}_∞≤\frac 18.

\begin{remark}
	\norm{B}_∞=\sup_{\norm{v}=1}\norm{Bv}_∞=\sup_{\norm{v}_{\R^N}=1}\max_{i=1,..,N}|(Bv)_i|=\sup_{\norm{v}_∞=1}\max_{i=1,...,N}|∑_{j=1}^nb_{ij}v_j}|≤\sup_{\norm{v}_∞=1}\max_{i=1,...,N}∑_{j=1}^N|b_{ij}||v_j|≤\sup_{\norm{v}_∞=1}\max_{i=1,...,N}∑_{j=1}^N∑_{j=1}^N|b_{ij}|\norm{v}_∞=\max_{i=1,...,N}∑_{j=1}^N|b_{ij}|
	
	En fait on a égalité: \norm{B}_∞=\max_{i=1,...,N}∑_{j=1}^N|b_{ij}. 
\end{remark}

	\norm{A_{h}\dmo}_∞=\max_{i=1,...,N}∑_{j=1}^N(A_h\dmo)_{ij}≤\max_{i=1,..,N}∑_{j=1}^N(A_{0h}\dmo)_{ij}=\norm{A_{0h}\dmo}_∞=\max∑_{j=1}^N(A_{0h}\dmo)_{ij}\times 1=\max_{i=1,...,N}A_{0h}\dmo e)_i=\norm{A_{0h}\dmo e\}_∞ où e=\vc{1\\\vdots\\1}.
	
	Soit d=A_{0h}\dmo e i.e. e=A_{0h}d (**). Si on discretise
	
	\[\left\{\mqty{-u''=1, (0,1)\\u(0)=u(1)=0}}\right.\]
	
	On arrive à (**) (i.e. ici b_h=\vc{1\\\vdots\\1}=e et U_h=d ici la solution exacte u est donne par μ(x)=\frac{x(1-x)}{2} => u^{(4)}=0 et donc la solution (U_h=d) coincide avec le vecteur (u(x_i))_{i=1,...,N} d_i=u(x_i)=u(ih)=\frac{ih(1-ih)}{2}\ \forall i=1,...,N)
	
	Donc \norm{d}_∞=\max_{i=1,...,N}|d_i|=\max_{i=1,...,N}(\frac{ih(1-ih)}2)≤\sup_{x\in[0,1]}\frac{x(1-x)}2=\frac18=μ(\frac 12).
\end{proof}
\section{Erreur de discretisation en $x_i$} % (fold)
\label{sec:erreur_de_discretisation_en_x_i}
e_i=u(x_i)-u_i, \forall i=1,...,N.

\begin{theorem}
	Soit u solution exacte de -u''+c(x)u=f(x),  (0,1) u(0)=u(1)=0. Supposons u\in C^{(4)}([0,1]). Soit U_h la solution de probleme discretisé (A_hU_h=b_h). Alors \max_{i=1,...,N}|e_i|≤\frac1{96}\norm{u^{4}}_∞h^2.
\end{theorem}
\begin{proof}
	on a (\tilde U_h=\vc{u(x_1)\\u(x_2)\\\vdots\\u(x_N)}) \norm{\tilde U_h-U_h}_∞=\max_{i=1,...,N}|(\tilde U_h-U_h)_i|=\max_{i=1,...N}|e_i| or A_h(\tilde U_h-U_h)=R (erreur de consistance).
	
	Donc: \norm{\tilde U_h-U_h}_∞=\norm{A_h\dmo R}_∞≤\norm{A_h\dmo}_∞\norm{R}_∞≤\frac18\frac1{12}\  ?\sup_{t\in(0,1)}|u^{(4)}(t)|h^2?
\end{proof}

\begin{exercise}
	Cas des conditions de Neumann. Écrire (SN) - scheme numérique - pour -u''+c(x)u=f \ch (0,1) u'(0)=α u'(1)=β 
	
	À titre d'exercise, traiter le cas:
	\[\left\{\mqty{-u''+c(x)u=f ch (0,1) \\ u(0)=0\\u'(1)=β}\right.\]
	
	
		\begin{system}
	
			\frac1{h^2}(2u_i-u_{i-1}-u_{i+1}+c_ih^2u_i)=f_i\ i=1,...,N\\
			\frac{u_1-u_0}h=α \\
			\frac{u_{N+1}-u_{N}}h=β
		\end{system}
	
	Ecriture matricielle de (SN) : 
	
	(SN) <=> A_hU_h=B_h
	
	U_{h}=\vc{u_0\\u_1\\\vdots\\u_N\\u_{N+1}}\in\R^{N+2}
	B_h=\vc{α\\b_1\\\vdots\\b_n\\β}\in\R^{N+2}.
	
	A_h\in M_{N+2}(\R) avec
	
	A_h=\mqty{
	-\frac 1h & \frac1h                   & 0                          &    &   & \cdots & 0                         & 0       \\
-1        & 2+c\_1h\textasciicircum 2 & 0                          &    &   & \cdots &                           &         \\
0         & -1                        & 2+c\_2\textasciicircum 2-1 & -1 & 0 &        &                           &         \\
          &                           &                            &    &   &        &                           &         \\
          &                           &                            &    &   & -1     & 2+c\_Nh\textasciicircum 2 & -1      \\
          &                           &                            & 0  & 0 & 0      & -\frac1h                  & \frac1h
}\in M_{N+2}(\R)

Eorme eliminée u_0 et u_{N+1}

i=1 => \frac1{h^2}(2u_1-u_0-u_2+c_1h^2u_1)=f_1 => \frac{u_0}{h^2}=\frac1{h^2}(2u_1-u_2+c_1h^2u_1)-b_1

\frac{u_0}{h}=\frac1h(2u_1-u_2+c_1h^2u_1)-hb_1

d'où (1) <=> \frac{u_1}{h}-\frac1h(2u_1-u_2+c_1h^2u_1)+hf_1=α

<=> (1+c_1h^2)u_1-u_2=h^2f_1-αh on elimine u_{N+1} des (2):

i=N => \frac1{h^2}(2u_N-u_{N-1}-u_{N+1}+c_Nh^2u_N)=b_N

=> u_{N+1}=(2u_N-u_{N-1}+c_Nh^2u_N)-h^2f_N

\frac1{h^2}(2u_i-u_{i-1}-u_{i+1}+c_ih^2u_i)=f_i, i=1,...,N

<=> -u_{N-1}-u_N+c_Nh^2u_n=-βh+h^2f_N

U_h=\vc{u_1\\u_2\\\vdots\\u_N}, B_h=\vc{h^2f_1-αh\\b_1\\\vdots\\b_{N-1}\\βh+h^2b_N}

Ici A_h=\mqty{
1+c\_1h\textasciicircum 2      & -1                                                               & 0                                                                & \cdots &    \\
-\frac1\{h\textasciicircum 2\} & \frac2\{h\textasciicircum 2\}+c\_2-\frac1\{h\textasciicircum 2\} & 0                                                                & \cdots &    \\
0                              & -\frac1\{h\textasciicircum 2\}                                   & \frac2\{h\textasciicircum 2\}+c\_3-\frac1\{h\textasciicircum 2\} &        &    \\
                               &                                                                  &                                                                  &        &    \\
}
\end{exercise}
% section erreur_de_discretisation_en_x_i (end)
\chapter{Formulation variateorinelle des problèmes aux limites} % (fold)
\label{cha:formulation_variateorinelle_des_problemes_aux_limites}

Introduction:
\left\{\mqty{-\delta u(x)=f(x)\text{ dans $Ω$ ouvert "regulier" de $\R^d$}\\u=0\text{ sur }\partial Ω}\right.

"f donnée" 

\delta u=∑_{i=1}^n\pdv[2]{u}{x_i}

-u''=f dans (0,1) u(0)=u(1)=0

-u''φ(x)=f(x)φ(x) => ∫_0^1-u''φ(x)=∫_0^1f(x)φ(x)

∫_0^1v'φ\dd{x}=[vφ]_0^1-∫_0^1vφ'\dd{x}
[-u'φ]^1_0+∫_0^1u'φ'\dd{x}=∫_0^1f(x)φ(x)\dd{x}

∫_0^1u'φ'\dd{x}=∫_0^1f(x)φ(x)\dd{x}\ \forall φ, φ(0)=φ(1)

f\in L^2(0,1)

(1) => ∫_Ω-\delta uφ\dd{x}=∫_Ωfφ\dd{x}.

Formule de Green: ∫_Ω-\delta uφ\dd{x}=∫_Ω\nabla u\nabla φ\dd{x}-∫_{\partial Ω}\pdv{u}{n}φ\dd{σ}.

\pdv{u}{n}(x)=\nabla n(x)•\vect n=\pdv{u}{x_1}(x)\cos(\vect n,x_1)+...+\pdv{u}{x_n}(x)\cos(\vect n, x_n)
-- Formule Green

Donc si φ=0 sur le bord \partial Ω.

(1) => ∫_Ω\nabla u\nabla φ\dd{x}=∫_Ωfφ\dd{x} \forall φ\in L^2(Ω), \nabla φ\in (L^2(Ω))^n

Definition
(2) s'appelle formulation faible de e'équation (1) et u solution de (2) est appelé solution faible de (1)

∫_0^1 u'φ'\dd{x}=\expval{-u'',φ}

Outils pour étudier la formulation faible: Espace Sobolev:

H^1(Ω):=\{u\in L^2(Ω), \nabla u\in(L^2(Ω))^n\}=\{u\in L^2(Ω),\pdv{u}{x_i}\in L^2(Ω)\ \forall i=1,...,n\}

H^2(Ω):=\{u\in L^2(Ω), \nabla u\in (H'(Ω))^n\}=\{u\in L^2(Ω), \pdv{u}{x_i}, \pdv{u}{x_i}{x_j}\in L^2(Ω), \forall i,j\}

H^m(Ω):=\{u\in L^2(Ω), \nabla u\in (H^{n-1}(Ω))^n\}

\begin{remark}
	H^0(Ω)=L^2(Ω)
\end{remark}

\begin{theorem}
	H^1(Ω) est un espace de Hilbert pour le problèmes defini par:
	a(u,v)=∫_Ωuv\dd{x}+∫_Ω\nabla u\nabla v\dd{x}=∫_Ωuv\dd{x}+∑_{i=1}^n∫_Ω\pdv{u}{x_i}\pdv{v}{x_i}\dd{x}
\end{theorem}

Norme associée au problème: \norm{u}_{H'(Ω)}=\sqrt{a(u,u)}=\sqrt{∫_Ωu^2\dd{x}+∑_i∫_Ω(\pdv{u}{x_i})^2\dd{x}}

\begin{proof}
	Montrons que H^2(Ω) est complet pour cette norme. 
\end{proof}
Soit (u_m)_m \in H^1(Ω) de Cauchy. \forall ε>0 \exists n_0\in\N: n,p≥n_0 => \norm{u_n-u_m}^2_{H^1}≤ε => ∫_Ω(u_n-u_m)^2\dd{x}+∫_Ω|\nabla u_n-\nabla u_m|^2\dd{x}\to0, n,m\to∞

<=> ∫_Ω(u_n-u_m)^2\dd{x}\to 0. n,p\to ∞. et ∫_Ω|\nabla u_n-\nabla u_m|^2\dd{x}\to 0, n,p\to ∞.

=> (u_m)_m et (\nabla u_m)_m est de Cauchy dans L^2(Ω) et (L^2(Ω)).

Comme L^2(Ω) est complet, \exists u\in L^2(Ω) t.q. u_m\to u dans L^2(Ω) c-à-d ∫_Ω|u_m-u|^2\dd{x}\to 0, m\to ∞. \existsξ\in (L^2(Ω))^n

∫_Ω|\nabla u_m-ξ|^2\dd{x}\to 0, m\to ∞.

u_m\to u dans L^2(Ω) => \pdv{u_m}{x_i}\to\pdv{u}{x_i} dans D'(Ω) (au sens des distributions)

\pdv{u_m}{x_i}\fto \pdv{u}{x_i} dans D'(Ω) <=> ∫_Ω\pdv{u_m}{x_i}φ\dd{x} -> ∫_Ω\pdv{u}{x_i}φ\dd{x}, \forall φ\in D(Ω).

D(Ω)=C_0^∞(Ω)=\{u\in C^∞(Ω), \supp u= K\subsetΩ, K -compact\}

Montrons (*):
∫_Ω\pdv{u_m}{x_i}φ\dd{x}=-∫_Ωu_m\pdv{φ}{x_i}\dd{x}\to-∫_Ωu\pdv{φ}{x_i}\dd{x}\ \forall i

car |-∫_Ωu_m\pdv{φ}{x_i}+∫_Ωu\pdv{φ}{x_i}\dd{x}|=|∫_Ω(u-u_m)\pdv{φ}{x_i}\dd{x}|≤(∫_Ω(u-u_m)^2\dd{x})^{\frac 12}(∫_Ω(\pdv{φ}{x_i})^2\dd{x})^{\frac12}\to 0.

\begin{remark}
	t_m\to t dans L^(Ω) => t_m\fto t dans D'(Ω).
	
	T\in D'(Ω) <=> T forme lineaire continue sur D(Ω).
	
	T:D(Ω)->\R
	φ|->T(φ)=\expval{T,φ}
		
	Si T=f\in L^2(Ω)\ \expval{T,φ}=∫_Ωfφ\dd{x}
	
	\expval{T,φ_1+φ_2}=∫_Ωf(φ_1+φ_2)\dd{x}
	
	φ_m\to φ, n\to ∞ <=> \sup_{x\in Ω}|φ_n(x)-φ(x)|\to 0 
	
	\sup_{x\in Ω}|D^αφ_n-D^αφ|\to 0
	m\to∞
	
	φ_m\to φ sur D(Ω) => T(φ_m)=∫_Ωf φ_n\dd{x}\to T(φ)=∫_Ωfφ\dd{x}
	
	|∫_Ωfφ\dd{x}-∫_Ωfφ_n\dd{x}|≤\norm{f}_{L^2(Ω)}\norm{φ-φ_n}_{L^2(Ω)}
	
	\norm{φ-φ_m}^2_{L^2(Ω)}=∫_Ω|φ-φ_m|^2\dd{x}≤\sup_{x\in Ω}|φ-φ_m|∫_Ω\dd{x}
	
	\nabla u_m\to ξ dans (L^2(Ω))^n donc \pdv{u_m}{x_i}\to ξ_i dans L^2(Ω) donc \pdv{u_m}{x_i} \fto ξ_i, D'(Ω) d'où ξ_i=\pdv{u}{x_i}\in L^2(Ω) \forall i=1,...,n et donc u\in H^1(Ω).
	
\end{remark}

\begin{definition}
	H_0^1(Ω)=\overline{D(Ω)} dans H^1(Ω). De u\in H_0^1(Ω), \exists(u_p)\in D(Ω) top \norm{u-u_p}_{H'(Ω)}\to0, p\to∞.
\end{definition}

Autre caractérisation de H_0^1(Ω): H_0^1(Ω)=\{u\in H'(Ω), u=0 sur \partial Ω\}
% chapter formulation_variateorinelle_des_problemes_aux_limites (end)



% MISSED missed 2 TD or lectures...


% section rappels_sur_le_fonctions_convexes_en_dimension_1 (end)
% chapter conditions_d_optimalite (end)

% start approximation lexture SILI Ali 12 Février (February) (2 lectures)

% end approximation lexture SILI Ali 12 Février (February)




